\section{Тест производительности}

Тест представлял из себя сравнение поразрядной сортировки, реализованной мной в этой лабораторной работе с STL сртировкой. Задачей была сортировка файла, состоящего из 10000 пар \enquote{ключ-значение}, и их упорядочивание по возрастанию ключа.

\begin{alltt}
lunidep@lunidep-VirtualBox:~/Desktop/da_lab1$ g++ -pedantic -std=c++17 -Wall -Werror da_lab1.cpp -o da_lab1
lunidep@lunidep-VirtualBox:~/Desktop/da_lab1$ ./da_lab1 < tests/01.t > tmp
lunidep@lunidep-VirtualBox:~/Desktop/da_lab1$ cat tmp | grep "time"
Radix sort time: 81956us
STL Sort time: 15318us
\end{alltt}

Как видно, что поразрядная сортировка выиграла у STL, засчет сравнительо небольшого объема сортируемых данных, при увеличении этого объема со временем STL сортировка начнет показывать себя более эффективной.



Также хочется продемонстрировать примерную линейность времени сортировки
    
    
    n – количество пар «ключ-значение» во входном файле
\begin{enumerate}
    \item n = 10; time = 0,002s
    \item n = 100; time = 0,002s
    \item n = 1000; time = 0,014s
    \item n = 10000; time = 0,192s
    \item n = 100000; time = 1,070s
\end{enumerate}
\pagebreak
