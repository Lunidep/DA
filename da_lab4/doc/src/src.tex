\section{Описание}

Интернет-ресурс \cite{wikipedia_BM} дает следующее описание алгоритму Бойера-Мура:\newline

Алгоритм основан на трёх идеях.

1. \textbf{Сканирование слева направо, сравнение справа налево}\newline
Совмещается начало текста (строки) и шаблона, проверка начинается с последнего символа шаблона. Если символы совпадают, производится сравнение предпоследнего символа шаблона и т. д. Если все символы шаблона совпали с наложенными символами строки, значит, подстрока найдена, и выполняется поиск следующего вхождения подстроки.\newline

Если же какой-то символ шаблона не совпадает с соответствующим символом строки, шаблон сдвигается на несколько символов вправо, и проверка снова начинается с последнего символа.\newline

Эти «несколько», вычисляются по двум эвристикам.\newline

2. \textbf{Эвристика стоп-символа}\newline
(Замечание: эвристика стоп-символа присутствует в большинстве описаний алгоритма Бойера — Мура, включая оригинальную статью Бойера и Мура, но не является необходимой для достижения оценки {\displaystyle O(n+m)}{\displaystyle O(n+m)} времени работы; более того, данная эвристика для своей работы требует дополнительной памяти и дополнительного времени на этапе подготовки шаблона.)\newline

3. \textbf{Эвристика совпавшего суффикса}\newline
Неформально, если при чтении шаблона справа налево совпал суффикс S, а символ b, стоящий перед S в шаблоне (то есть шаблон имеет вид PbS), не совпал, то эвристика совпавшего суффикса сдвигает шаблон на наименьшее число позиций вправо так, чтобы строка S совпала с шаблоном, а символ, предшествующий в шаблоне данному совпадению S, отличался бы от b (если такой символ вообще есть). Формально, для данного шаблона s[0..m{-}1] считается целочисленный массив suffshift[0..m], в котором suffshift[i] равно минимальному числу j>0, такому что s[i{-}j] != s[i{-}1] (если i>0 и i-j >= 0) и s[i{-}j{+}k] = s[i{-}1{+}k] для любого k>0, для которого выполняется 0 <= i-j+k<m и 0 <= i-1+k<m (для пояснения смотрите примеры ниже). Затем, если при чтении шаблона s справа налево совпало k{-}1 символов s[m{-}1],s[m{-}2], ... ,s[m{-}k{+}1], а символ s[m{-}k] не совпал, то шаблон сдвигается на suffshift[m-k] символов вправо. 

\pagebreak

\section{Исходный код}

Основная логика работы прогарммы состоит в следующем: двигаясь по входному тексту слева направо(начиная с символа, по индексу равному размеру входного паттерна), идет его сравнение с паттерном справаа налево.\newline 

При выявлении несоответствия, проискодит сдвиг по входному тексту на параметр $offset$.
Данный параметр вычисляется с помощью двух эвристик, описанных выше. Применяется та эвристика, которая разрешит нам больший сдвиг.\newline

Если паттерн пройден полностью, значит зафиксировано вхождение подстроки. И программа печатает номер строки и номер индекса в этой строке, где было зафиксировано вхождение.\newline

Остановлюсь подробнее на подсчете сдвигов.\newline
Код выисления \textbf{эвристики плохого символа}:\newline
\begin{lstlisting}[language=C]
class TBadLetterRule {
public:
	TBadLetterRule(const vector<uint32_t>& pattern) :
		patt_size(pattern.size()) {
		for (int i = patt_size - 1; i >= 0; --i) {
			rule[pattern[i]].push_back(i);
		}
	}

	int Use(uint32_t letter, size_t idx_patt) const {
		auto it = rule.find(letter);
		if (it == rule.end()) {
			return patt_size - idx_patt;
		}
		const vector<uint32_t>& idxs = it->second;
		for (uint32_t i : idxs) {
			if (i > idx_patt){
				return 1;
			}
			else if (i < idx_patt) {
				return idx_patt - i;
			}
		}
		return patt_size;
	}

private:
	unordered_map<uint32_t, vector<uint32_t>> rule;
	size_t patt_size;
};
\end{lstlisting}

Код выисления \textbf{эвристики хорошего суффикса}
\begin{lstlisting}[language=C]
class TGoodSuffix {
public:
	TGoodSuffix(const vector<uint32_t>& pattern) : patt_size(pattern.size())
	{
		rule = LFunction(pattern, gp_size);
	}

	int Use(size_t idx_patt) const {
		if (idx_patt >= patt_size) {
			return 1;
		}
		if (rule[idx_patt] == UNDEFINED) {
			return patt_size - gp_size;
		}
		return patt_size - rule[idx_patt] - 1;
	}

	size_t Getgp_size() const {
		return gp_size;
	}
private:
	vector<size_t> rule;
	size_t patt_size;
	size_t gp_size = 0;
};
\end{lstlisting}

Особого внимания заслуживает способ вычисления L-функции:
\begin{lstlisting}[language=C]
vector<size_t> ZFunction(const vector<uint32_t>& pattern) {
	size_t n = pattern.size();
	vector<size_t> res(n, 0);
	size_t l = 0;
	size_t r = 0;
	for (size_t i = 1; i < n; ++i) {
		if (i <= r) {
			res[i] = min(r - i + 1, res[i - l]);
		}
		while (i + res[i] < n && pattern[res[i]] == pattern[i + res[i]]) {
			++res[i];
		}
		if (i + res[i] - 1 > r) {
			l = i;
			r = i + res[i] - 1;
		}
	}
	return res;
}

vector<size_t> NFunction(vector<uint32_t> pattern) {
	reverse(pattern.begin(), pattern.end());
	size_t n = pattern.size();
	vector<size_t> z_func = ZFunction(move(pattern));
	vector<size_t> res(n);
	for (size_t i = 0; i < n; ++i) {
		res[i] = z_func[n - i - 1];
	}
	return res;
}

const size_t UNDEFINED = -1;

vector<size_t> LFunction(const vector<uint32_t>& pattern, size_t& gp_size) {
	gp_size = 0;
	size_t n = pattern.size();
	vector<size_t> res(n, UNDEFINED);
	vector<size_t> n_func = NFunction(move(pattern));
	for (size_t i = 0; i < n; ++i) {
		size_t j = n - n_func[i];
		if (j != n) {
			res[j] = i;
			if (i == n - j - 1) {
				gp_size = i + 1;
			}
		}
	}
	return res;
}
\end{lstlisting}

\section{Консоль}
tmp:
\begin{alltt}
11 45 11 45 90
0011 45 011 0045 11 45 90    11
45 11 45 90
\end{alltt}

console output:
\begin{alltt}
1, 3
1, 8
\end{alltt}
\pagebreak