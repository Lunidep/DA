\section{Тест производительности}

Тест представлял из себя сравнение реализованного мной алгорима Бойера-Мура c бинарным поиском, реализованным в STL посредством функции upper_bound.\newline

В результате работы $benchmark.cpp$ видны следующие результаты:

\begin{alltt}
root@Lunidep:~/DA/da_lab4# ./wrapper.sh 
[info][Tue May 17 11:35:24 MSK 2022] Stage #1. Compiling...
g++ -std=c++17 -pedantic -Wall -Wextra -Wno-unused-variable da_lab4.cpp -o da_lab4
g++ -std=c++17 -pedantic -Wall -Wextra -Wno-unused-variable benchmark.cpp -o benchmark
[info][Tue May 17 11:35:25 MSK 2022] Compiling OK
[info][Tue May 17 11:35:25 MSK 2022] Stage #2. Benchmark generating...
[info][Tue May 17 11:35:26 MSK 2022] Benchmark generating OK
[info][Tue May 17 11:35:26 MSK 2022] Stage #3. Benchmark results:
BM_search: 181 ms
find_search: 619 ms
[info][Tue May 17 11:35:28 MSK 2022] Benchmark OK
\end{alltt}

Из примера видно, релизованный мной алгорим превосходит своего STL-соперника, потому что он обладает линейной сложностью, а бинарный поиск - операция за O(log(n)).


\pagebreak
