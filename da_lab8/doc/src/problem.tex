\CWHeader{Лабораторная работа \textnumero 8}

\CWProblem{
Бычкам дают пищевые добавки, чтобы ускорить их рост. Каждая добавка содержит некоторые из N действующих веществ. Соотношения количеств веществ в добавках могут отличаться.

Воздействие добавки определяется как $c_{1}a_{1} + c_{2}a_{2} + ... + c_{N}a_{N}$, где $a_{i}$ - количество i-го вещества в добавке, $c_{i}$ – неизвестный коэффициент, связанный с веществом и не зависящий от добавки. Чтобы найти неизвестные коэффициенты $c_{i}$, Биолог может измерить воздействие любой добавки, использовав один её мешок. Известна цена мешка каждой из M $(M \leqslant N)$ различных добавок. Нужно помочь Биологу подобрать самый дешевый наобор добавок, позволяющий найти коэффициенты $c_{i}$. Возможно, соотношения веществ в добавках таковы, что определить коэффициенты нельзя.\newline

\textbf{Формат ввода}\newline
В первой строке текста – целые числа M и N; в каждой из следующих M строк записаны N чисел, задающих соотношение количеств веществ в ней, а за ними – цена мешка добавки. Порядок веществ во всех описаниях добавок один и тот же, все числа – неотрицательные целые не больше 50.\newline

\textbf{Формат вывода}\newline 
Вывести -1 если определить коэффциенты невозможно, иначе набор добавок (и их номеров по порядоку во входных данных). Если вариантов несколько, вывести какой-либо из них.\newline
}
\pagebreak
