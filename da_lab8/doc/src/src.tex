\section{Описание}

\textbf{Жадный алгоритм} — алгоритм, заключающийся в принятии локально оптимальных решений на каждом этапе, допуская, что конечное решение также окажется оптимальным. Известно, что если структура задачи задается матроидом, тогда применение жадного алгоритма выдаст глобальный оптимум.

Если глобальная оптимальность алгоритма имеет место практически всегда, его обычно предпочитают другим методам оптимизации, таким как динамическое программирование.\newline
\pagebreak

\section{Исходный код}
Фрагмент кода, где применяются рекуррентные формулы:\newline


\begin{lstlisting}[language=C++]
for (int i = 0; i < N; ++i) {
        int minimal_price = DEFAULT_MINIMAL_PRICE; //недостижимая цена
        int row = DEFAULT_VALUE; //строка с минимальной стоимостью
        int mtx_line_number = DEFAULT_VALUE; //номер считанной строки
        int price = N;
        for (int j = i; j < M; ++j) {
            if (mtx[j][i] != 0 && mtx[j][price] < minimal_price) {
                minimal_price = mtx[j][price];
                row = j;
            }
        }
        if (row == -1) {
            cout << -1 << endl;
            return 0;
        }
        mtx_line_number = mtx[row][price + 1];
        answer.push_back(mtx_line_number);
        mtx[i].swap(mtx[row]);
        for (int k = i + 1; k < M; ++k) {
            double m = mtx[k][i] / mtx[i][i];
            for (int v = i; v < N; ++v) {
                mtx[k][v] -= mtx[i][v] * m;
            }
        }
    }

\end{lstlisting}

\section{Консоль}
tmp:
\begin{alltt}
3 3
1 0 2 3
1 0 2 4
2 0 1 2
\end{alltt}

console output:
\begin{alltt}
-1
\end{alltt}
\pagebreak