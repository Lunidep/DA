\section{Выводы}

Алгоритм A* является потомком алгорима Дейкстры, который в силу своей жадности обладает таким недостатком, как выбор неоптимального пути до финишной вершины.\newline

Алгоритм A* и допустим, и обходит при этом минимальное количество вершин, благодаря тому, что он работает с «оптимистичной» оценкой пути через вершину. Оптимистичной в том смысле, что, если он пойдёт через эту вершину, алгоритм уверен, что реальная стоимость результата будет равна этой оценке, но никак не меньше.\newline

Когда A* завершает поиск, он, согласно определению, нашёл путь, истинная стоимость которого меньше, чем оценка стоимости любого пути через любой открытый узел. Но поскольку эти оценки являются оптимистичными, соответствующие узлы можно без сомнений отбросить. Иначе говоря, A* никогда не упустит возможности минимизировать длину пути, и потому является допустимым.\newline

Недостатком алгоритма А* является необходимость придумать эвристику. Однако, во многих задачах эта эвристика находится достаточно легко и естественно. \newline

\pagebreak
