\section{Описание}

\textbf{Поиск A*} (произносится «А звезда» или «А стар», от англ. A star) — в информатике и математике, алгоритм поиска по первому наилучшему совпадению на графе, который находит маршрут с наименьшей стоимостью от одной вершины (начальной) к другой (целевой, конечной).\newline

Порядок обхода вершин определяется эвристической функцией «расстояние + стоимость» (обычно обозначаемой как f(x)). Эта функция — сумма двух других: функции стоимости достижения рассматриваемой вершины (x) из начальной (обычно обозначается как g(x) и может быть как эвристической, так и нет), и функции эвристической оценки расстояния от рассматриваемой вершины к конечной (обозначается как h(x)).\newline

Функция h(x) должна быть допустимой эвристической оценкой, то есть не должна переоценивать расстояния к целевой вершине. Например, для задачи маршрутизации h(x) может представлять собой расстояние до цели по прямой линии, так как это физически наименьшее возможное расстояние между двумя точками.\newline

Этот алгоритм был впервые описан в 1968 году Питером Хартом, Нильсом Нильсоном и Бертрамом Рафаэлем. Это по сути было расширение алгоритма Дейкстры, созданного в 1959 году. Новый алгоритм достигал более высокой производительности (по времени) с помощью эвристики. В их работе он упоминается как «алгоритм A». Но так как он вычисляет лучший маршрут для заданной эвристики, он был назван A*.\newline
\pagebreak

\section{Исходный код}

\begin{lstlisting}[language=C++]
#include <bits/stdc++.h>

using namespace std;
using ll = long long; 

#define endl '\n';

struct coords {
    double x;
    double y;
};

inline double euclid_dist(ll u, ll v, const vector<coords>& vert) {
    coords v1 = vert[u];
    coords v2 = vert[v];
    
    return sqrt((v1.x - v2.x) * (v1.x - v2.x) + (v1.y - v2.y) * (v1.y - v2.y));
}

struct node {
    ll ind;
    double path;
    
    node(const ll _ind, const double _path) {
        ind = _ind;
        path = _path;
    }

    friend bool operator< (const node& v1, const node& v2) {
        if (v1.path != v2.path)
            return v1.path > v2.path;
        return v1.ind < v2.ind;
    }
};

double solve(const ll start, const ll finish, const vector<coords> &vert, const vector<vector<ll>> &graph) {
    vector<double> res(vert.size(), DBL_MAX);//используем вектор расстояний в том числе, как вектор посещенных вершин
    vector<double> dist(vert.size(), DBL_MAX);
    priority_queue<node> pq;

    res[start] = 0; //расстояние до стартовой вершины
    dist[start] = res[start] + euclid_dist(start, finish, vert);
    pq.push(node(start, dist[start]));
    
    while (!pq.empty()) {
        node tmp = pq.top();
        pq.pop();

        if (tmp.ind == finish)//конечная вершина
            break;

        if (tmp.path > dist[tmp.ind])
            continue;

        for (auto next : graph[tmp.ind]){//соседи рассматриваемой вершины

            if (res[next] == DBL_MAX || (res[tmp.ind] + euclid_dist(tmp.ind, next, vert)) < res[next]){
                res[next] = res[tmp.ind] + euclid_dist(tmp.ind, next, vert);
                dist[next] = res[next] + euclid_dist(next, finish, vert);

                pq.push(node(next, dist[next]));
            }
        }
    }
    return res[finish];
}

int main() {
    ios_base::sync_with_stdio(false);
    cin.tie(nullptr); cout.tie(nullptr);

    ll n, m;
    cin >> n >> m;
    vector<coords> vert(n);
    vector<vector<ll>> graph(n);

    for (ll i = 0; i < n; ++i)
        cin >> vert[i].x >> vert[i].y;

    for (ll i = 0; i < m; ++i) {
        ll u, v;
        cin >> u >> v;
        u--; v--;
        graph[u].push_back(v);
        graph[v].push_back(u);
    }

    ll q;
    cin >> q;

    while(q--) {
        ll u, v;
        cin >> u >> v;
        u--; v--;
        double ans = solve(u, v, vert, graph);
        if (ans != DBL_MAX){
            cout << fixed << setprecision(6) << ans << endl;
        }
        else{
            cout << -1 << endl;
        }
    }
    return 0;
}
\end{lstlisting}

\section{Консоль}
console input:
\begin{alltt}
4 5
0 0
1 1
-1 1
0 2
1 2
1 3
2 4
3 4
1 4
1
1 4

\end{alltt}
console output:
\begin{alltt}
2.000000
\end{alltt}
\pagebreak

\section{Оценка сложности}
Временная сложность алгоритма A* зависит от эвристики. В худшем случае, число вершин, исследуемых алгоритмом, растёт экспоненциально по сравнению с длиной оптимального пути, но сложность становится полиномиальной, когда эвристика удовлетворяет следующему условию:\newline

${\displaystyle |h(x)-h^{*}(x)|\leq O(\log h^{*}(x))}|h(x)-h^{*}(x)|\leq O(\log h^{*}(x))$;\newline
где h* — оптимальная эвристика, то есть точная оценка расстояния из вершины x к цели. Другими словами, ошибка h(x) не должна расти быстрее, чем логарифм от оптимальной эвристики.\newline

Но ещё бо́льшую проблему, чем временна́я сложность, представляют собой потребляемые алгоритмом ресурсы памяти. В худшем случае ему приходится помнить экспоненциальное количество узлов. \newline

\pagebreak