\CWHeader{Курсовой проект}

\CWProblem{
Реализуйте алгоритм A* для графа на решетке.\newline

\textbf{Формат ввода}\newline
В первой строке вам даны два числа n и m (1 \leqslant n \leqslant ${10^4}$, 1 \leqslant m \leqslant ${10^5}$) — количество вершин и рёбер в графе. \newline

В следующих n строках вам даны пары чисел x y(${-10^9}$ \leqslant x, y \leqslant ${10^9}$), описывающие положение вершин графа в двумерном пространстве. \newline
В следующих m строках даны пары чисел в отрезке от 1 до n, описывающие рёбра графа.\newline

Далее дано число q $(1 \leqslant q \leqslant 300)$ и в следующих q строках даны запросы в виде пар чисел a b $(1 \leqslant a, b \leqslant n)$ на поиск кратчайшего пути между двумя вершнами.\newline

\textbf{Формат вывода}\newline 

В ответ на каждый запрос выведите единственное число — длину кратчайшего пути между заданными вершинами с абсолютной либо относительной точностью ${10^{-6}}$, если пути между вершинами не существует выведите {--1}.
}
\pagebreak
