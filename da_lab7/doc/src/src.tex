\section{Описание}

\textbf{Динамическое программирование}\newline в теории управления и теории вычислительных систем — способ решения сложных задач путём разбиения их на более простые подзадачи. Он применим к задачам с оптимальной подструктурой, выглядящим как набор перекрывающихся подзадач, сложность которых чуть меньше исходной. В этом случае время вычислений, по сравнению с «наивными» методами, можно значительно сократить.\newline

Ключевая идея в динамическом программировании достаточно проста. Как правило, чтобы решить поставленную задачу, требуется решить отдельные части задачи (подзадачи), после чего объединить решения подзадач в одно общее решение. Часто многие из этих подзадач одинаковы. Подход динамического программирования состоит в том, чтобы решить каждую подзадачу только один раз, сократив тем самым количество вычислений. Это особенно полезно в случаях, когда число повторяющихся подзадач экспоненциально велико.\newline

Метод динамического программирования сверху — это простое запоминание результатов решения тех подзадач, которые могут повторно встретиться в дальнейшем. Динамическое программирование снизу включает в себя переформулирование сложной задачи в виде рекурсивной последовательности более простых подзадач.\newline
\pagebreak

\section{Исходный код}
Фрагмент кода, где применяются рекуррентные формулы:\newline


\begin{lstlisting}[language=C]
for (int i = 2; i <= n; i++) {
    coast[i] = coast[i - 1] + i; //вычитание единицы
    res[i] = 1;

    if (i % 2 == 0 && coast[i / 2] + i < coast[i]) { //деление на 2
        coast[i] = coast[i / 2] + i;
        res[i] = 2;
    }
    if (i % 3 == 0 && coast[i / 3] + i < coast[i]) { //деление на 3
        coast[i] = coast[i / 3] + i;
        res[i] = 3;
    }

}    
\end{lstlisting}

\section{Консоль}
tmp:
\begin{alltt}
82
\end{alltt}

console output:
\begin{alltt}
202
-1 /3 /3 /3 /3
\end{alltt}
\pagebreak