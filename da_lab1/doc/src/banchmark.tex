\section{Тест производительности}

Тест представлял из себя сравнение поразрядной сортировки, реализованной мной в этой лабораторной работе с STL сртировкой. Задачей была сортировка файла, состоящего из 1 000 000 пар \enquote{ключ-значение}, и их упорядочивание по возрастанию ключа.

\begin{alltt}
lunidep@lunidep-VirtualBox:~/Desktop/da_lab1$ ./wrapper.sh
[info][Вс 20 мар 2022 00:26:23 MSK] Stage #1. Compiling...
g++ -std=c++14 -pedantic -Wall -Wextra -Wno-unused-variable sort.o da_lab1.cpp -o da_lab1
[info][Вс 20 мар 2022 00:26:23 MSK] Compiling OK
[info][Вс 20 мар 2022 00:26:23 MSK] Stage #2. Test generating...
[info][Вс 20 мар 2022 00:26:23 MSK] Test generating OK
[info][Вс 20 мар 2022 00:26:23 MSK] Stage #3. Chacking...
[info][Вс 20 мар 2022 00:26:23 MSK] tests/01.t, lines = 51 OK
[info][Вс 20 мар 2022 00:26:23 MSK] tests/02.t, lines = 981 OK
[info][Вс 20 мар 2022 00:26:23 MSK] tests/03.t, lines = 978 OK
[info][Вс 20 мар 2022 00:26:23 MSK] tests/04.t, lines = 443 OK
[info][Вс 20 мар 2022 00:26:23 MSK] tests/05.t, lines = 891 OK
[info][Вс 20 мар 2022 00:26:23 MSK] tests/06.t, lines = 794 OK
[info][Вс 20 мар 2022 00:26:23 MSK] tests/07.t, lines = 464 OK
[info][Вс 20 мар 2022 00:26:23 MSK] tests/08.t, lines = 597 OK
[info][Вс 20 мар 2022 00:26:23 MSK] tests/09.t, lines = 23 OK
[info][Вс 20 мар 2022 00:26:23 MSK] tests/10.t, lines = 765 OK
[info][Вс 20 мар 2022 00:26:23 MSK] Checking OK
[info][Вс 20 мар 2022 00:26:23 MSK] Stage #4 Benchmark test generating...
[info][Вс 20 мар 2022 00:26:42 MSK] Benchmark test generating OK
[info][Вс 20 мар 2022 00:26:42 MSK] Stage #5 Benchmarking...
make: 'benchmark' is up to date.
[info][Вс 20 мар 2022 00:26:42 MSK] Running benchmark.t
\textbf{Count of lines is 1000000
Counting sort time: 8801623us
STL Sort time: 2180026us}
[info][Вс 20 мар 2022 00:26:53 MSK] Benchmarking OK
\end{alltt}

Из примера видно, что при больших объемах данных поразрядная сортировка проигрывает STL, однако при меньших объемах, она показывет себя более эффективной.

Также хочется продемонстрировать примерную линейность времени сортировки n – количество пар «ключ-значение» во входном файле
\begin{enumerate}
    \item n = 10; time = 0,002s
    \item n = 100; time = 0,002s
    \item n = 1000; time = 0,014s
    \item n = 10000; time = 0,192s
    \item n = 100000; time = 1,070s
    \item n = 1000000; time = 14,106s
\end{enumerate}
\pagebreak
