\section{Выводы}

Выполнив первую лабораторную работу по курсу \enquote{Дискретный анализ}, мною были изучены и реализованы два алгоритма сортировки за линейное время – поразрядная сортировка и сортировка подсчётом.

Сортировка за линейное время наиболее эффективна при обработке небольшого количества данных. Сложность сортировок этого типа – O(d*n), где d – количество разрядов, по которым происходит сортировка, n – объём входных данных. Для сортировки большого количества данных этот тип сортировок будет неэффективен.

Также стоит отметить устойчивость линейных сортировок – элементы с одинаковыми ключами не меняют порядок в отсортированном наборе данных.

\pagebreak