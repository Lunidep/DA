\section{Выводы}

Выполнив девятую лабораторную работу по курсу «Дискретный анализ», я больше
узнал о графах и алгоритмах работы с ними. Теория графов к настоящему вре-
мени содержит достаточно много эффективных инструментов для решения столь
же широкого круга проблем. Однако, средства и идеи, сегодня относящиеся к обла-
сти дискретной математики, именуемой теорией графов, пребывают по сей момент
в некотором единстве.\newline

Так, в решаемой мной задаче , можно заметить, что граф неориентированный, и мож-
но использовать обход в глубину. При запуске обхода из вершины, принадлежащей к
некоторой компоненте связности, обход посетит все вершины из этой компоненты и
только их. Таким образом, в функцию обхода можно передавать вектор, в который
будут помещаться вершины из очередной компоненты связности.\newline

Сложность совпадает со сложностью обхода в глубину, то есть O(V + E).\newline

Существуют и другие алгоритмы поиска кратчайшего пути от одной вершины графа
до другой. Например, алгоритм Форда-Беллмана. Он имеет сложность O(V * E), но умеет работать с ребрами, имеющими отрицательный вес. Алгоритм Дейкстры же является жадным и имеет сложность O(VlogV).\newline

\pagebreak
