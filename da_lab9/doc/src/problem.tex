\CWHeader{Лабораторная работа \textnumero 9}

\CWProblem{
Задан взвешенный неориентированный граф, состоящий из n вершин и m ребер. Вершины пронумерованы целыми числами от 1 до n. Необходимо найти длину кратчайшего пути из вершины с номером start в вершину с номером finish при помощи алгоритма Дейкстры. Длина пути равна сумме весов ребер на этом пути. Граф не содержит петель и кратных ребер.\newline

\textbf{Формат ввода}\newline
В первой строке заданы 1 \leqslant n  \leqslant {10^5} и 1 \leqslant m  \leqslant {10^5}, 1 \leqslant start \leqslant n и 1 \leqslant finish \leqslant n. В следующих m строках записаны ребра. Каждая строка содержит три числа – номера вершин, соединенных ребром, и вес данного ребра. Вес ребра – целое число от 0 до {10^9}.\newline

\textbf{Формат вывода}\newline 
Необходимо вывести одно число – длину кратчайшего пути между указанными вершинами. Если пути между указанными вершинами не существует, следует вывести строку "No solution" (без кавычек).\newline
}
\pagebreak
