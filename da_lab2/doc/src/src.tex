\section{Описание}

Интернет-ресурс \cite{kvodo_avl_tree} дает следующее описание AVL-деревьям:

АВЛ-дерево – структура данных, изобретенная в 1968 году двумя советскими математиками: Евгением Михайловичем Ландисом и Георгием Максимовичем Адельсон-Вельским. Прежде чем дать конструктивное определение АВЛ-дереву, сделаем это для сбалансированного двоичного дерева поиска.\newline

Сбалансированным называется такое двоичное дерево поиска, в котором высота каждого из поддеревьев, имеющих общий корень, отличается не более чем на некоторую константу k, и при этом выполняются условия характерные для двоичного дерева поиска.\newline

АВЛ-дерево – сбалансированное двоичное дерево поиска с k=1. Для его узлов определен коэффициент сбалансированности (balance factor). Balance factor – это разность высот правого и левого поддеревьев, принимающая одно значение из множества {-1, 0, 1}. Ниже изображен пример АВЛ-дерева, каждому узлу которого поставлен в соответствие его реальный коэффициент сбалансированности.\newline

Сбалансированное дерево эффективно в обработке, что следует из следующих рассуждений. Максимальное количество шагов, которое может потребоваться для обнаружения нужного узла, равно количеству уровней самого бинарного дерева поиска. А так как поддеревья сбалансированного дерева, \enquote{растущие} из произвольного корня, практически симметричны, то и его листья расположены на сравнительно невысоком уровне, т. е. высота дерева сводиться к оптимальному минимуму. Поэтому критерий баланса положительно сказывается на общей производительности. Но в процессе обработки АВЛ-дерева, балансировка может нарушиться, тогда потребуется осуществить операцию балансировки. Помимо нее, над АВЛ-деревом определены операции вставки и удаления элемента. Именно выполнение последних может привести к дисбалансу дерева.\newline

Доказано, что высота АВЛ-дерева, имеющего N узлов, примерно равна log2N. Имея в виду это, а также то, то, что время выполнения операций добавления и удаления напрямую зависит от операции поиска, получим временную сложность трех операций для худшего и среднего случая  – O(logN).\newline

\pagebreak

\section{Исходный код}

Программа разделена на три файла:
\begin{enumerate}
    \item $main.cpp$
    \item $avl.h$
    \item $detail avl.h$
\end{enumerate}

В файле $main.cpp$ описан только интерфейс программы, основная логика же описана в двух других файлах.\newline

В файле $avl.cpp$ описана структура узла моего дерева:
\begin{lstlisting}[language=C]
struct TAvlNode {
	K key;
	V value;
	unsigned long long height;
	TAvlNode *left;
	TAvlNode *right;
	TAvlNode() : key(), value(), height{1}, left{nullptr}, 
						right{nullptr} {};
	TAvlNode(K k, V v) : key{k}, value{v}, height{1}, 
						left{nullptr}, right{nullptr} {};
};
\end{lstlisting}


Реализованный класс AVL-дерева(TAvl) с основными методами: $Add$, $Delete$, $Find$.
Но наиболее интереснвя честь реализации AVL-дерева скрывается внутри них, а именно балансировка после вставки и удаления элеметов.\newline

Относительно АВЛ-дерева балансировкой вершины называется операция, которая в случае разницы высот левого и правого поддеревьев = 2, изменяет связи предок-потомок в поддереве данной вершины так, что разница становится <= 1, иначе ничего не меняет. Указанный результат получается вращениями поддерева данной вершины.\newline

Реализована данная механика путем совершением деревом 4 видов различных поворотов: \cite{wikipedia_avl_tree}:
\begin{enumerate}
    \item Малое левое вращение AVL LR.
Данное вращение используется тогда, когда высота b-поддерева — высота L = 2 и высота С <= высота R.
    \item Большое левое вращение AVL BR.
Данное вращение используется тогда, когда высота b-поддерева — высота L = 2 и высота c-поддерева > высота R.
    \item Малое правое вращение AVL LL.
Данное вращение используется тогда, когда высота b-поддерева — высота R = 2 и высота С <= высота L.
    \item Большое правое вращение AVL BL.
Данное вращение используется тогда, когда высота b-поддерева — высота R = 2 и высота c-поддерева > высота L.
\end{enumerate}

В каждом случае достаточно просто доказать то, что операция приводит к нужному результату и что полная высота уменьшается не более чем на 1 и не может увеличиться. Также можно заметить, что большое вращение это комбинация правого и левого малого вращения. Из-за условия балансированности высота дерева О(log(N)), где N- количество вершин, поэтому добавление элемента требует O(log(N)) операций.\newline


\begin{lstlisting}[language=C]
ull Height(const TNode *node) {
		return node != nullptr ? node->height : 0;
	}

	int Balance(const TNode *node) {
		return Height(node->left) - Height(node->right);
	}

	void Reheight(TNode *node) {
		node->height = std::max(Height(node->left), Height(node->right)) + 1;
	}

	TNode *RotateLeft(TNode *a) {
		TNode *b = a->right;
		a->right = b->left;
		b->left = a;
		Reheight(a);
		Reheight(b);
		return b;
	}

	TNode *RotateRight(TNode *a) {
		TNode *b = a->left;
		a->left = b->right;
		b->right = a;
		Reheight(a);
		Reheight(b);
		return b;
	}

	TNode *RLrotate(TNode *a) {
		a->right = RotateRight(a->right);
		return RotateLeft(a);
	}

	TNode *LRrotate(TNode *a) {
		a->left = RotateLeft(a->left);
		return RotateRight(a);
	}

	TNode *Rebalance(TNode *node) {
		if (node == nullptr) {
			return nullptr;
		}
		Reheight(node);
		int balanceRes = Balance(node);
		if (balanceRes == -2) {
			if (Balance(node->right) == 1) {
				return RLrotate(node);
			}
			return RotateLeft(node);
		}
		else if (balanceRes == 2) {
			if (Balance(node->left) == -1) {
				return LRrotate(node);
			}
			return RotateRight(node);
		}
		return node;
	}
\end{lstlisting}

В файле $detailavl.h$ описан алгоритм сохранения и загрузки из бинарного файла объектов моего класса $TDetailAvl$, который являтся наследником ранее опианного класса $TAvl$. Именно с этим классом взаимодействует пользователь.\newline


\section{Консоль}
tmp:
\begin{alltt}
+ a 1
+ A 2
+ aaaaaaaaaaaaaaaaaaaaaaaaaaaaaaaaaaaaaaaaaaaaaaaaaaaaaaaaaaaaaaaaaaaaaaaaaaaaaaaaaaaaaaaaaaaaaaaaaaaaaaaaaaaaaaaaaaaaaaaaaaaaaaaaaaaaaaaaaaaaaaaaaaaaaaaaaaaaaaaaaaaaaaaaaaaaaaaaaaaaaaaaaaaaaaaaaaaaaaaaaaaaaaaaaaaaaaaaaaaaaaaaaaaaaaaaaaaaaaaaaaaaaaaaaaaaaaaa 18446744073709551615
aaaaaaaaaaaaaaaaaaaaaaaaaaaaaaaaaaaaaaaaaaaaaaaaaaaaaaaaaaaaaaaaaaaaaaaaaaaaaaaaaaaaaaaaaaaaaaaaaaaaaaaaaaaaaaaaaaaaaaaaaaaaaaaaaaaaaaaaaaaaaaaaaaaaaaaaaaaaaaaaaaaaaaaaaaaaaaaaaaaaaaaaaaaaaaaaaaaaaaaaaaaaaaaaaaaaaaaaaaaaaaaaaaaaaaaaaaaaaaaaaaaaaaaaaaaaaaaa
A
- A
a
\end{alltt}

console output:
\begin{alltt}
root@Lunidep:~/DA/da_lab2/src# ./da_lab2 < tmp
OK
Exist
OK
OK: 18446744073709551615
OK: 1
OK
NoSuchWord
\end{alltt}
\pagebreak