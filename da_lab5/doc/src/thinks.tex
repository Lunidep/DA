\section{Выводы}

Выполнив  пятую лабораторную работу по курсу \enquote{Дискретный анализ}, мною были изучены суффиксные деревья, а также различные виды их построения и использования для решения прикладных задач.\newline

С их помощью, обработав текст можно получить поиск сложностью О(длина текста), в то время, как у алгоритмов, разобранных в 4 лабораторной работе сложность поиска составляла О(длина паттерна). В случае достаточного количества памяти и понимания, что длины паттернов >> длина текста, предпочтительнее использовать Суффиксное дерево, нежели пользоваться другими способами поиска подстроки в строке.\newline

\pagebreak
